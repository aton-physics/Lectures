
% Default to the notebook output style

    


% Inherit from the specified cell style.




    
\documentclass[11pt]{article}

    
    
    \usepackage[T1]{fontenc}
    % Nicer default font (+ math font) than Computer Modern for most use cases
    \usepackage{mathpazo}

    % Basic figure setup, for now with no caption control since it's done
    % automatically by Pandoc (which extracts ![](path) syntax from Markdown).
    \usepackage{graphicx}
    % We will generate all images so they have a width \maxwidth. This means
    % that they will get their normal width if they fit onto the page, but
    % are scaled down if they would overflow the margins.
    \makeatletter
    \def\maxwidth{\ifdim\Gin@nat@width>\linewidth\linewidth
    \else\Gin@nat@width\fi}
    \makeatother
    \let\Oldincludegraphics\includegraphics
    % Set max figure width to be 80% of text width, for now hardcoded.
    \renewcommand{\includegraphics}[1]{\Oldincludegraphics[width=.8\maxwidth]{#1}}
    % Ensure that by default, figures have no caption (until we provide a
    % proper Figure object with a Caption API and a way to capture that
    % in the conversion process - todo).
    \usepackage{caption}
    \DeclareCaptionLabelFormat{nolabel}{}
    \captionsetup{labelformat=nolabel}

    \usepackage{adjustbox} % Used to constrain images to a maximum size 
    \usepackage{xcolor} % Allow colors to be defined
    \usepackage{enumerate} % Needed for markdown enumerations to work
    \usepackage{geometry} % Used to adjust the document margins
    \usepackage{amsmath} % Equations
    \usepackage{amssymb} % Equations
    \usepackage{textcomp} % defines textquotesingle
    % Hack from http://tex.stackexchange.com/a/47451/13684:
    \AtBeginDocument{%
        \def\PYZsq{\textquotesingle}% Upright quotes in Pygmentized code
    }
    \usepackage{upquote} % Upright quotes for verbatim code
    \usepackage{eurosym} % defines \euro
    \usepackage[mathletters]{ucs} % Extended unicode (utf-8) support
    \usepackage[utf8x]{inputenc} % Allow utf-8 characters in the tex document
    \usepackage{fancyvrb} % verbatim replacement that allows latex
    \usepackage{grffile} % extends the file name processing of package graphics 
                         % to support a larger range 
    % The hyperref package gives us a pdf with properly built
    % internal navigation ('pdf bookmarks' for the table of contents,
    % internal cross-reference links, web links for URLs, etc.)
    \usepackage{hyperref}
    \usepackage{longtable} % longtable support required by pandoc >1.10
    \usepackage{booktabs}  % table support for pandoc > 1.12.2
    \usepackage[inline]{enumitem} % IRkernel/repr support (it uses the enumerate* environment)
    \usepackage[normalem]{ulem} % ulem is needed to support strikethroughs (\sout)
                                % normalem makes italics be italics, not underlines
    

    
    
    % Colors for the hyperref package
    \definecolor{urlcolor}{rgb}{0,.145,.698}
    \definecolor{linkcolor}{rgb}{.71,0.21,0.01}
    \definecolor{citecolor}{rgb}{.12,.54,.11}

    % ANSI colors
    \definecolor{ansi-black}{HTML}{3E424D}
    \definecolor{ansi-black-intense}{HTML}{282C36}
    \definecolor{ansi-red}{HTML}{E75C58}
    \definecolor{ansi-red-intense}{HTML}{B22B31}
    \definecolor{ansi-green}{HTML}{00A250}
    \definecolor{ansi-green-intense}{HTML}{007427}
    \definecolor{ansi-yellow}{HTML}{DDB62B}
    \definecolor{ansi-yellow-intense}{HTML}{B27D12}
    \definecolor{ansi-blue}{HTML}{208FFB}
    \definecolor{ansi-blue-intense}{HTML}{0065CA}
    \definecolor{ansi-magenta}{HTML}{D160C4}
    \definecolor{ansi-magenta-intense}{HTML}{A03196}
    \definecolor{ansi-cyan}{HTML}{60C6C8}
    \definecolor{ansi-cyan-intense}{HTML}{258F8F}
    \definecolor{ansi-white}{HTML}{C5C1B4}
    \definecolor{ansi-white-intense}{HTML}{A1A6B2}

    % commands and environments needed by pandoc snippets
    % extracted from the output of `pandoc -s`
    \providecommand{\tightlist}{%
      \setlength{\itemsep}{0pt}\setlength{\parskip}{0pt}}
    \DefineVerbatimEnvironment{Highlighting}{Verbatim}{commandchars=\\\{\}}
    % Add ',fontsize=\small' for more characters per line
    \newenvironment{Shaded}{}{}
    \newcommand{\KeywordTok}[1]{\textcolor[rgb]{0.00,0.44,0.13}{\textbf{{#1}}}}
    \newcommand{\DataTypeTok}[1]{\textcolor[rgb]{0.56,0.13,0.00}{{#1}}}
    \newcommand{\DecValTok}[1]{\textcolor[rgb]{0.25,0.63,0.44}{{#1}}}
    \newcommand{\BaseNTok}[1]{\textcolor[rgb]{0.25,0.63,0.44}{{#1}}}
    \newcommand{\FloatTok}[1]{\textcolor[rgb]{0.25,0.63,0.44}{{#1}}}
    \newcommand{\CharTok}[1]{\textcolor[rgb]{0.25,0.44,0.63}{{#1}}}
    \newcommand{\StringTok}[1]{\textcolor[rgb]{0.25,0.44,0.63}{{#1}}}
    \newcommand{\CommentTok}[1]{\textcolor[rgb]{0.38,0.63,0.69}{\textit{{#1}}}}
    \newcommand{\OtherTok}[1]{\textcolor[rgb]{0.00,0.44,0.13}{{#1}}}
    \newcommand{\AlertTok}[1]{\textcolor[rgb]{1.00,0.00,0.00}{\textbf{{#1}}}}
    \newcommand{\FunctionTok}[1]{\textcolor[rgb]{0.02,0.16,0.49}{{#1}}}
    \newcommand{\RegionMarkerTok}[1]{{#1}}
    \newcommand{\ErrorTok}[1]{\textcolor[rgb]{1.00,0.00,0.00}{\textbf{{#1}}}}
    \newcommand{\NormalTok}[1]{{#1}}
    
    % Additional commands for more recent versions of Pandoc
    \newcommand{\ConstantTok}[1]{\textcolor[rgb]{0.53,0.00,0.00}{{#1}}}
    \newcommand{\SpecialCharTok}[1]{\textcolor[rgb]{0.25,0.44,0.63}{{#1}}}
    \newcommand{\VerbatimStringTok}[1]{\textcolor[rgb]{0.25,0.44,0.63}{{#1}}}
    \newcommand{\SpecialStringTok}[1]{\textcolor[rgb]{0.73,0.40,0.53}{{#1}}}
    \newcommand{\ImportTok}[1]{{#1}}
    \newcommand{\DocumentationTok}[1]{\textcolor[rgb]{0.73,0.13,0.13}{\textit{{#1}}}}
    \newcommand{\AnnotationTok}[1]{\textcolor[rgb]{0.38,0.63,0.69}{\textbf{\textit{{#1}}}}}
    \newcommand{\CommentVarTok}[1]{\textcolor[rgb]{0.38,0.63,0.69}{\textbf{\textit{{#1}}}}}
    \newcommand{\VariableTok}[1]{\textcolor[rgb]{0.10,0.09,0.49}{{#1}}}
    \newcommand{\ControlFlowTok}[1]{\textcolor[rgb]{0.00,0.44,0.13}{\textbf{{#1}}}}
    \newcommand{\OperatorTok}[1]{\textcolor[rgb]{0.40,0.40,0.40}{{#1}}}
    \newcommand{\BuiltInTok}[1]{{#1}}
    \newcommand{\ExtensionTok}[1]{{#1}}
    \newcommand{\PreprocessorTok}[1]{\textcolor[rgb]{0.74,0.48,0.00}{{#1}}}
    \newcommand{\AttributeTok}[1]{\textcolor[rgb]{0.49,0.56,0.16}{{#1}}}
    \newcommand{\InformationTok}[1]{\textcolor[rgb]{0.38,0.63,0.69}{\textbf{\textit{{#1}}}}}
    \newcommand{\WarningTok}[1]{\textcolor[rgb]{0.38,0.63,0.69}{\textbf{\textit{{#1}}}}}
    
    
    % Define a nice break command that doesn't care if a line doesn't already
    % exist.
    \def\br{\hspace*{\fill} \\* }
    % Math Jax compatability definitions
    \def\gt{>}
    \def\lt{<}
    % Document parameters
    \title{Assign3}
    
    
    

    % Pygments definitions
    
\makeatletter
\def\PY@reset{\let\PY@it=\relax \let\PY@bf=\relax%
    \let\PY@ul=\relax \let\PY@tc=\relax%
    \let\PY@bc=\relax \let\PY@ff=\relax}
\def\PY@tok#1{\csname PY@tok@#1\endcsname}
\def\PY@toks#1+{\ifx\relax#1\empty\else%
    \PY@tok{#1}\expandafter\PY@toks\fi}
\def\PY@do#1{\PY@bc{\PY@tc{\PY@ul{%
    \PY@it{\PY@bf{\PY@ff{#1}}}}}}}
\def\PY#1#2{\PY@reset\PY@toks#1+\relax+\PY@do{#2}}

\expandafter\def\csname PY@tok@w\endcsname{\def\PY@tc##1{\textcolor[rgb]{0.73,0.73,0.73}{##1}}}
\expandafter\def\csname PY@tok@c\endcsname{\let\PY@it=\textit\def\PY@tc##1{\textcolor[rgb]{0.25,0.50,0.50}{##1}}}
\expandafter\def\csname PY@tok@cp\endcsname{\def\PY@tc##1{\textcolor[rgb]{0.74,0.48,0.00}{##1}}}
\expandafter\def\csname PY@tok@k\endcsname{\let\PY@bf=\textbf\def\PY@tc##1{\textcolor[rgb]{0.00,0.50,0.00}{##1}}}
\expandafter\def\csname PY@tok@kp\endcsname{\def\PY@tc##1{\textcolor[rgb]{0.00,0.50,0.00}{##1}}}
\expandafter\def\csname PY@tok@kt\endcsname{\def\PY@tc##1{\textcolor[rgb]{0.69,0.00,0.25}{##1}}}
\expandafter\def\csname PY@tok@o\endcsname{\def\PY@tc##1{\textcolor[rgb]{0.40,0.40,0.40}{##1}}}
\expandafter\def\csname PY@tok@ow\endcsname{\let\PY@bf=\textbf\def\PY@tc##1{\textcolor[rgb]{0.67,0.13,1.00}{##1}}}
\expandafter\def\csname PY@tok@nb\endcsname{\def\PY@tc##1{\textcolor[rgb]{0.00,0.50,0.00}{##1}}}
\expandafter\def\csname PY@tok@nf\endcsname{\def\PY@tc##1{\textcolor[rgb]{0.00,0.00,1.00}{##1}}}
\expandafter\def\csname PY@tok@nc\endcsname{\let\PY@bf=\textbf\def\PY@tc##1{\textcolor[rgb]{0.00,0.00,1.00}{##1}}}
\expandafter\def\csname PY@tok@nn\endcsname{\let\PY@bf=\textbf\def\PY@tc##1{\textcolor[rgb]{0.00,0.00,1.00}{##1}}}
\expandafter\def\csname PY@tok@ne\endcsname{\let\PY@bf=\textbf\def\PY@tc##1{\textcolor[rgb]{0.82,0.25,0.23}{##1}}}
\expandafter\def\csname PY@tok@nv\endcsname{\def\PY@tc##1{\textcolor[rgb]{0.10,0.09,0.49}{##1}}}
\expandafter\def\csname PY@tok@no\endcsname{\def\PY@tc##1{\textcolor[rgb]{0.53,0.00,0.00}{##1}}}
\expandafter\def\csname PY@tok@nl\endcsname{\def\PY@tc##1{\textcolor[rgb]{0.63,0.63,0.00}{##1}}}
\expandafter\def\csname PY@tok@ni\endcsname{\let\PY@bf=\textbf\def\PY@tc##1{\textcolor[rgb]{0.60,0.60,0.60}{##1}}}
\expandafter\def\csname PY@tok@na\endcsname{\def\PY@tc##1{\textcolor[rgb]{0.49,0.56,0.16}{##1}}}
\expandafter\def\csname PY@tok@nt\endcsname{\let\PY@bf=\textbf\def\PY@tc##1{\textcolor[rgb]{0.00,0.50,0.00}{##1}}}
\expandafter\def\csname PY@tok@nd\endcsname{\def\PY@tc##1{\textcolor[rgb]{0.67,0.13,1.00}{##1}}}
\expandafter\def\csname PY@tok@s\endcsname{\def\PY@tc##1{\textcolor[rgb]{0.73,0.13,0.13}{##1}}}
\expandafter\def\csname PY@tok@sd\endcsname{\let\PY@it=\textit\def\PY@tc##1{\textcolor[rgb]{0.73,0.13,0.13}{##1}}}
\expandafter\def\csname PY@tok@si\endcsname{\let\PY@bf=\textbf\def\PY@tc##1{\textcolor[rgb]{0.73,0.40,0.53}{##1}}}
\expandafter\def\csname PY@tok@se\endcsname{\let\PY@bf=\textbf\def\PY@tc##1{\textcolor[rgb]{0.73,0.40,0.13}{##1}}}
\expandafter\def\csname PY@tok@sr\endcsname{\def\PY@tc##1{\textcolor[rgb]{0.73,0.40,0.53}{##1}}}
\expandafter\def\csname PY@tok@ss\endcsname{\def\PY@tc##1{\textcolor[rgb]{0.10,0.09,0.49}{##1}}}
\expandafter\def\csname PY@tok@sx\endcsname{\def\PY@tc##1{\textcolor[rgb]{0.00,0.50,0.00}{##1}}}
\expandafter\def\csname PY@tok@m\endcsname{\def\PY@tc##1{\textcolor[rgb]{0.40,0.40,0.40}{##1}}}
\expandafter\def\csname PY@tok@gh\endcsname{\let\PY@bf=\textbf\def\PY@tc##1{\textcolor[rgb]{0.00,0.00,0.50}{##1}}}
\expandafter\def\csname PY@tok@gu\endcsname{\let\PY@bf=\textbf\def\PY@tc##1{\textcolor[rgb]{0.50,0.00,0.50}{##1}}}
\expandafter\def\csname PY@tok@gd\endcsname{\def\PY@tc##1{\textcolor[rgb]{0.63,0.00,0.00}{##1}}}
\expandafter\def\csname PY@tok@gi\endcsname{\def\PY@tc##1{\textcolor[rgb]{0.00,0.63,0.00}{##1}}}
\expandafter\def\csname PY@tok@gr\endcsname{\def\PY@tc##1{\textcolor[rgb]{1.00,0.00,0.00}{##1}}}
\expandafter\def\csname PY@tok@ge\endcsname{\let\PY@it=\textit}
\expandafter\def\csname PY@tok@gs\endcsname{\let\PY@bf=\textbf}
\expandafter\def\csname PY@tok@gp\endcsname{\let\PY@bf=\textbf\def\PY@tc##1{\textcolor[rgb]{0.00,0.00,0.50}{##1}}}
\expandafter\def\csname PY@tok@go\endcsname{\def\PY@tc##1{\textcolor[rgb]{0.53,0.53,0.53}{##1}}}
\expandafter\def\csname PY@tok@gt\endcsname{\def\PY@tc##1{\textcolor[rgb]{0.00,0.27,0.87}{##1}}}
\expandafter\def\csname PY@tok@err\endcsname{\def\PY@bc##1{\setlength{\fboxsep}{0pt}\fcolorbox[rgb]{1.00,0.00,0.00}{1,1,1}{\strut ##1}}}
\expandafter\def\csname PY@tok@kc\endcsname{\let\PY@bf=\textbf\def\PY@tc##1{\textcolor[rgb]{0.00,0.50,0.00}{##1}}}
\expandafter\def\csname PY@tok@kd\endcsname{\let\PY@bf=\textbf\def\PY@tc##1{\textcolor[rgb]{0.00,0.50,0.00}{##1}}}
\expandafter\def\csname PY@tok@kn\endcsname{\let\PY@bf=\textbf\def\PY@tc##1{\textcolor[rgb]{0.00,0.50,0.00}{##1}}}
\expandafter\def\csname PY@tok@kr\endcsname{\let\PY@bf=\textbf\def\PY@tc##1{\textcolor[rgb]{0.00,0.50,0.00}{##1}}}
\expandafter\def\csname PY@tok@bp\endcsname{\def\PY@tc##1{\textcolor[rgb]{0.00,0.50,0.00}{##1}}}
\expandafter\def\csname PY@tok@fm\endcsname{\def\PY@tc##1{\textcolor[rgb]{0.00,0.00,1.00}{##1}}}
\expandafter\def\csname PY@tok@vc\endcsname{\def\PY@tc##1{\textcolor[rgb]{0.10,0.09,0.49}{##1}}}
\expandafter\def\csname PY@tok@vg\endcsname{\def\PY@tc##1{\textcolor[rgb]{0.10,0.09,0.49}{##1}}}
\expandafter\def\csname PY@tok@vi\endcsname{\def\PY@tc##1{\textcolor[rgb]{0.10,0.09,0.49}{##1}}}
\expandafter\def\csname PY@tok@vm\endcsname{\def\PY@tc##1{\textcolor[rgb]{0.10,0.09,0.49}{##1}}}
\expandafter\def\csname PY@tok@sa\endcsname{\def\PY@tc##1{\textcolor[rgb]{0.73,0.13,0.13}{##1}}}
\expandafter\def\csname PY@tok@sb\endcsname{\def\PY@tc##1{\textcolor[rgb]{0.73,0.13,0.13}{##1}}}
\expandafter\def\csname PY@tok@sc\endcsname{\def\PY@tc##1{\textcolor[rgb]{0.73,0.13,0.13}{##1}}}
\expandafter\def\csname PY@tok@dl\endcsname{\def\PY@tc##1{\textcolor[rgb]{0.73,0.13,0.13}{##1}}}
\expandafter\def\csname PY@tok@s2\endcsname{\def\PY@tc##1{\textcolor[rgb]{0.73,0.13,0.13}{##1}}}
\expandafter\def\csname PY@tok@sh\endcsname{\def\PY@tc##1{\textcolor[rgb]{0.73,0.13,0.13}{##1}}}
\expandafter\def\csname PY@tok@s1\endcsname{\def\PY@tc##1{\textcolor[rgb]{0.73,0.13,0.13}{##1}}}
\expandafter\def\csname PY@tok@mb\endcsname{\def\PY@tc##1{\textcolor[rgb]{0.40,0.40,0.40}{##1}}}
\expandafter\def\csname PY@tok@mf\endcsname{\def\PY@tc##1{\textcolor[rgb]{0.40,0.40,0.40}{##1}}}
\expandafter\def\csname PY@tok@mh\endcsname{\def\PY@tc##1{\textcolor[rgb]{0.40,0.40,0.40}{##1}}}
\expandafter\def\csname PY@tok@mi\endcsname{\def\PY@tc##1{\textcolor[rgb]{0.40,0.40,0.40}{##1}}}
\expandafter\def\csname PY@tok@il\endcsname{\def\PY@tc##1{\textcolor[rgb]{0.40,0.40,0.40}{##1}}}
\expandafter\def\csname PY@tok@mo\endcsname{\def\PY@tc##1{\textcolor[rgb]{0.40,0.40,0.40}{##1}}}
\expandafter\def\csname PY@tok@ch\endcsname{\let\PY@it=\textit\def\PY@tc##1{\textcolor[rgb]{0.25,0.50,0.50}{##1}}}
\expandafter\def\csname PY@tok@cm\endcsname{\let\PY@it=\textit\def\PY@tc##1{\textcolor[rgb]{0.25,0.50,0.50}{##1}}}
\expandafter\def\csname PY@tok@cpf\endcsname{\let\PY@it=\textit\def\PY@tc##1{\textcolor[rgb]{0.25,0.50,0.50}{##1}}}
\expandafter\def\csname PY@tok@c1\endcsname{\let\PY@it=\textit\def\PY@tc##1{\textcolor[rgb]{0.25,0.50,0.50}{##1}}}
\expandafter\def\csname PY@tok@cs\endcsname{\let\PY@it=\textit\def\PY@tc##1{\textcolor[rgb]{0.25,0.50,0.50}{##1}}}

\def\PYZbs{\char`\\}
\def\PYZus{\char`\_}
\def\PYZob{\char`\{}
\def\PYZcb{\char`\}}
\def\PYZca{\char`\^}
\def\PYZam{\char`\&}
\def\PYZlt{\char`\<}
\def\PYZgt{\char`\>}
\def\PYZsh{\char`\#}
\def\PYZpc{\char`\%}
\def\PYZdl{\char`\$}
\def\PYZhy{\char`\-}
\def\PYZsq{\char`\'}
\def\PYZdq{\char`\"}
\def\PYZti{\char`\~}
% for compatibility with earlier versions
\def\PYZat{@}
\def\PYZlb{[}
\def\PYZrb{]}
\makeatother


    % Exact colors from NB
    \definecolor{incolor}{rgb}{0.0, 0.0, 0.5}
    \definecolor{outcolor}{rgb}{0.545, 0.0, 0.0}



    
    % Prevent overflowing lines due to hard-to-break entities
    \sloppy 
    % Setup hyperref package
    \hypersetup{
      breaklinks=true,  % so long urls are correctly broken across lines
      colorlinks=true,
      urlcolor=urlcolor,
      linkcolor=linkcolor,
      citecolor=citecolor,
      }
    % Slightly bigger margins than the latex defaults
    
    \geometry{verbose,tmargin=1in,bmargin=1in,lmargin=1in,rmargin=1in}
    
    

    \begin{document}
    
    
    \maketitle
    
    

    
    Assignment 3

\hypertarget{github-user-name-aton-physics}{%
\subsection{GitHub User Name:
aton-physics}\label{github-user-name-aton-physics}}

\hypertarget{email-andrew_tonbrown.edu}{%
\subsection{Email:
andrew\_ton@brown.edu}\label{email-andrew_tonbrown.edu}}

    \begin{Verbatim}[commandchars=\\\{\}]
{\color{incolor}In [{\color{incolor}4}]:} \PY{c+c1}{\PYZsh{} Keep import statement in their own cell at the top}
        \PY{k+kn}{import} \PY{n+nn}{matplotlib}\PY{n+nn}{.}\PY{n+nn}{pyplot} \PY{k}{as} \PY{n+nn}{plt} \PY{c+c1}{\PYZsh{} for plotting          }
        \PY{k+kn}{import} \PY{n+nn}{numpy} \PY{k}{as} \PY{n+nn}{np}
        
        \PY{c+c1}{\PYZsh{} import our Pendulum class from Particle1D}
        \PY{k+kn}{from} \PY{n+nn}{Particle1D} \PY{k}{import} \PY{n}{Pendulum}
        \PY{o}{\PYZpc{}}\PY{k}{matplotlib} inline  
\end{Verbatim}


    \hypertarget{problem-1}{%
\subsection{Problem 1}\label{problem-1}}

\begin{enumerate}
\def\labelenumi{\alph{enumi})}
\item
  Calculate the phase difference between two pendulums nearby in the
  chaotic regime.
\item
  Plot log of phase difference versus time, qualitatively noting the
  slope of this semilog plot as the Lyapunov Exponent.
\end{enumerate}

    \begin{Verbatim}[commandchars=\\\{\}]
{\color{incolor}In [{\color{incolor}2}]:} \PY{c+c1}{\PYZsh{} Driving force amplitude}
        \PY{n}{Fd} \PY{o}{=} \PY{l+m+mf}{1.1}
        
        \PY{c+c1}{\PYZsh{} Initial theta}
        \PY{n}{theta0} \PY{o}{=} \PY{l+m+mf}{0.2}
        
        \PY{c+c1}{\PYZsh{} Difference in initial position}
        \PY{n}{delta\PYZus{}theta0} \PY{o}{=} \PY{l+m+mf}{0.01}
        
        
        \PY{n}{tf} \PY{o}{=} \PY{l+m+mi}{200}
        \PY{n}{dt} \PY{o}{=} \PY{l+m+mf}{0.01}
        
        \PY{c+c1}{\PYZsh{} Initialize objects and calculate trajectories}
        \PY{n}{p1} \PY{o}{=} \PY{n}{Pendulum}\PY{p}{(}\PY{n}{nu} \PY{o}{=} \PY{l+m+mi}{1}\PY{o}{/}\PY{l+m+mi}{2}\PY{p}{,} \PY{n}{Fd} \PY{o}{=} \PY{n}{Fd}\PY{p}{,} \PY{n}{omega\PYZus{}d} \PY{o}{=} \PY{l+m+mf}{0.6} \PY{p}{,} \PY{n}{x0} \PY{o}{=} \PY{n}{theta0}\PY{p}{,} \PY{n}{tf} \PY{o}{=} \PY{n}{tf}\PY{p}{,} \PY{n}{dt} \PY{o}{=} \PY{n}{dt}\PY{p}{)}
        \PY{n}{p2} \PY{o}{=} \PY{n}{Pendulum}\PY{p}{(}\PY{n}{nu} \PY{o}{=} \PY{l+m+mi}{1}\PY{o}{/}\PY{l+m+mi}{2}\PY{p}{,} \PY{n}{Fd} \PY{o}{=} \PY{n}{Fd}\PY{p}{,} \PY{n}{omega\PYZus{}d} \PY{o}{=} \PY{l+m+mf}{0.6} \PY{p}{,} \PY{n}{x0} \PY{o}{=} \PY{n}{theta0}\PY{o}{+}\PY{n}{delta\PYZus{}theta0}\PY{p}{,} \PY{n}{tf} \PY{o}{=} \PY{n}{tf}\PY{p}{,} \PY{n}{dt} \PY{o}{=} \PY{n}{dt}\PY{p}{)}
        \PY{n}{p1}\PY{o}{.}\PY{n}{scipy\PYZus{}trajectory}\PY{p}{(}\PY{p}{)}
        \PY{n}{p2}\PY{o}{.}\PY{n}{scipy\PYZus{}trajectory}\PY{p}{(}\PY{p}{)}
        
        \PY{c+c1}{\PYZsh{} difference between two directorys}
        \PY{n}{delta\PYZus{}theta} \PY{o}{=} \PY{n}{np}\PY{o}{.}\PY{n}{abs}\PY{p}{(}\PY{n}{p1}\PY{o}{.}\PY{n}{xv\PYZus{}unwrap}\PY{p}{[}\PY{p}{:}\PY{p}{,}\PY{l+m+mi}{0}\PY{p}{]} \PY{o}{\PYZhy{}} \PY{n}{p2}\PY{o}{.}\PY{n}{xv\PYZus{}unwrap}\PY{p}{[}\PY{p}{:}\PY{p}{,}\PY{l+m+mi}{0}\PY{p}{]}\PY{p}{)}
        
        \PY{c+c1}{\PYZsh{} Make Plot    }
        \PY{n}{fig} \PY{o}{=} \PY{n}{plt}\PY{o}{.}\PY{n}{figure}\PY{p}{(}\PY{n}{figsize} \PY{o}{=} \PY{p}{[}\PY{l+m+mi}{8}\PY{p}{,}\PY{l+m+mi}{6}\PY{p}{]}\PY{p}{)}
        \PY{n}{ax} \PY{o}{=} \PY{n}{fig}\PY{o}{.}\PY{n}{subplots}\PY{p}{(}\PY{l+m+mi}{1}\PY{p}{,}\PY{l+m+mi}{1}\PY{p}{)}
        
        \PY{c+c1}{\PYZsh{} plot on a log scale}
        \PY{n}{ax}\PY{o}{.}\PY{n}{semilogy}\PY{p}{(}\PY{n}{p1}\PY{o}{.}\PY{n}{tarray}\PY{p}{,} \PY{n}{delta\PYZus{}theta}\PY{p}{,} \PY{n}{label} \PY{o}{=} \PY{l+s+s2}{\PYZdq{}}\PY{l+s+s2}{\PYZdl{}F\PYZus{}d =\PYZdl{} }\PY{l+s+si}{\PYZob{}\PYZcb{}}\PY{l+s+s2}{\PYZdq{}}\PY{o}{.}\PY{n}{format}\PY{p}{(}\PY{n}{Fd}\PY{p}{)}\PY{p}{)}
        
        \PY{n}{ax}\PY{o}{.}\PY{n}{set\PYZus{}ylabel}\PY{p}{(}\PY{l+s+s1}{\PYZsq{}}\PY{l+s+s1}{\PYZdl{}}\PY{l+s+s1}{\PYZbs{}}\PY{l+s+s1}{delta }\PY{l+s+se}{\PYZbs{}\PYZbs{}}\PY{l+s+s1}{theta\PYZdl{} (rad)}\PY{l+s+s1}{\PYZsq{}}\PY{p}{,} \PY{n}{fontsize} \PY{o}{=} \PY{l+m+mi}{14}\PY{p}{)}
        \PY{n}{ax}\PY{o}{.}\PY{n}{set\PYZus{}xlabel}\PY{p}{(}\PY{l+s+s1}{\PYZsq{}}\PY{l+s+s1}{t (s)}\PY{l+s+s1}{\PYZsq{}}\PY{p}{,} \PY{n}{fontsize} \PY{o}{=} \PY{l+m+mi}{14}\PY{p}{)}
        \PY{n}{ax}\PY{o}{.}\PY{n}{legend}\PY{p}{(}\PY{n}{loc} \PY{o}{=} \PY{l+m+mi}{4}\PY{p}{,} \PY{n}{fontsize} \PY{o}{=} \PY{l+m+mi}{14}\PY{p}{)}
            
        \PY{n}{fig}\PY{o}{.}\PY{n}{subplots\PYZus{}adjust}\PY{p}{(}\PY{n}{top} \PY{o}{=} \PY{l+m+mf}{0.96}\PY{p}{,} \PY{n}{bottom} \PY{o}{=}\PY{l+m+mf}{0.08}\PY{p}{,} \PY{n}{left} \PY{o}{=} \PY{l+m+mf}{0.24}\PY{p}{,} \PY{n}{right} \PY{o}{=}\PY{l+m+mf}{0.98}\PY{p}{)}
        \PY{n}{plt}\PY{o}{.}\PY{n}{grid}\PY{p}{(}\PY{p}{)}
        \PY{n}{plt}\PY{o}{.}\PY{n}{plot}\PY{p}{(}\PY{p}{[}\PY{l+m+mi}{0}\PY{p}{,} \PY{l+m+mi}{30}\PY{p}{]}\PY{p}{,} \PY{p}{[}\PY{l+m+mi}{10}\PY{o}{*}\PY{o}{*}\PY{o}{\PYZhy{}}\PY{l+m+mi}{2}\PY{p}{,} \PY{l+m+mi}{10}\PY{p}{]}\PY{p}{,} \PY{l+s+s1}{\PYZsq{}}\PY{l+s+s1}{k\PYZhy{}}\PY{l+s+s1}{\PYZsq{}}\PY{p}{,} \PY{n}{lw}\PY{o}{=}\PY{l+m+mi}{2}\PY{p}{)}
        \PY{n}{plt}\PY{o}{.}\PY{n}{show}\PY{p}{(}\PY{p}{)}
\end{Verbatim}


    \begin{center}
    \adjustimage{max size={0.9\linewidth}{0.9\paperheight}}{output_3_0.png}
    \end{center}
    { \hspace*{\fill} \\}
    
    Recall that the lyapunov exponent appears in the following relation:
\(log(\delta\theta) \sim \lambda t\) so to extract it from the data we
can simply plot the phase difference between the pendulums on a semi-log
plot. If one can identify the linear regime of this semilog plot, one
can simply read off the lyapunov exponent for a qualitative estimate.
Here I count a slope of about 3, so the lyapunov exponent is given by
\(\lambda \approx\) 3 Hz (units come from demanding \(\lambda t\) is
dimensionless)

    \hypertarget{problem-2}{%
\subsection{Problem 2}\label{problem-2}}

\begin{enumerate}
\def\labelenumi{\alph{enumi})}
\tightlist
\item
  Calculate Poincare sections of the pendulum as it undergoes
  period-doubling to chaos
\end{enumerate}

things to note: Plot \(\omega\) versus \(\theta\) with one point plotted
for each drive cycle for Driving force 1.4, 1.44, 1.465, 1.481, 1.49
with the parameters of the problem.

The attractor in the period-1 regime only contains a single point.
Remove points from the initial transient.

Extract points for the poincare section small enough with respect to the
step size.

    \begin{Verbatim}[commandchars=\\\{\}]
{\color{incolor}In [{\color{incolor}187}]:} \PY{k+kn}{import} \PY{n+nn}{importlib}
          \PY{k+kn}{import} \PY{n+nn}{Particle1D}
          \PY{k+kn}{from} \PY{n+nn}{Particle1D} \PY{k}{import} \PY{n}{Pendulum}
          \PY{n}{importlib}\PY{o}{.}\PY{n}{reload}\PY{p}{(}\PY{n}{Particle1D}\PY{p}{)}
\end{Verbatim}


\begin{Verbatim}[commandchars=\\\{\}]
{\color{outcolor}Out[{\color{outcolor}187}]:} <module 'Particle1D' from '/home/jovyan/Lectures/Assignments/Particle1D.py'>
\end{Verbatim}
            
    \begin{Verbatim}[commandchars=\\\{\}]
{\color{incolor}In [{\color{incolor}188}]:} \PY{n}{Fd} \PY{o}{=} \PY{p}{[}\PY{l+m+mf}{1.4}\PY{p}{,} \PY{l+m+mf}{1.44}\PY{p}{,} \PY{l+m+mf}{1.465}\PY{p}{,} \PY{l+m+mf}{1.481}\PY{p}{,} \PY{l+m+mf}{1.49}\PY{p}{]}
          \PY{n}{theta0} \PY{o}{=} \PY{l+m+mf}{0.2}
          \PY{n}{tf} \PY{o}{=} \PY{l+m+mi}{1000}
          \PY{n}{dt} \PY{o}{=} \PY{l+m+mf}{0.005}
          
          \PY{c+c1}{\PYZsh{} Initialize objects and calculate trajectories}
          \PY{k}{for} \PY{n}{it} \PY{o+ow}{in} \PY{n+nb}{range}\PY{p}{(}\PY{n+nb}{len}\PY{p}{(}\PY{n}{Fd}\PY{p}{)}\PY{p}{)}\PY{p}{:}
              \PY{n}{p1} \PY{o}{=} \PY{n}{Pendulum}\PY{p}{(}\PY{n}{nu} \PY{o}{=} \PY{l+m+mi}{1}\PY{o}{/}\PY{l+m+mi}{2}\PY{p}{,} \PY{n}{Fd} \PY{o}{=} \PY{n}{Fd}\PY{p}{[}\PY{n}{it}\PY{p}{]}\PY{p}{,} \PY{n}{omega\PYZus{}d} \PY{o}{=} \PY{l+m+mi}{2}\PY{o}{/}\PY{l+m+mi}{3} \PY{p}{,} \PY{n}{x0} \PY{o}{=} \PY{n}{theta0}\PY{p}{,} \PY{n}{tf} \PY{o}{=} \PY{n}{tf}\PY{p}{,} \PY{n}{dt} \PY{o}{=} \PY{n}{dt}\PY{p}{)}
              \PY{n}{p1}\PY{o}{.}\PY{n}{scipy\PYZus{}trajectory}\PY{p}{(}\PY{p}{)}
              \PY{n+nb}{print}\PY{p}{(}\PY{l+s+s1}{\PYZsq{}}\PY{l+s+s1}{done}\PY{l+s+s1}{\PYZsq{}}\PY{p}{)}
              \PY{n}{p1}\PY{o}{.}\PY{n}{plot}\PY{p}{(}\PY{n}{pt} \PY{o}{=} \PY{l+s+s1}{\PYZsq{}}\PY{l+s+s1}{phase}\PY{l+s+s1}{\PYZsq{}}\PY{p}{)}
              \PY{c+c1}{\PYZsh{}p1.plot(pt = \PYZsq{}trajectory\PYZsq{})}
\end{Verbatim}


    \begin{Verbatim}[commandchars=\\\{\}]
done
done
done
done
done

    \end{Verbatim}

    \begin{center}
    \adjustimage{max size={0.9\linewidth}{0.9\paperheight}}{output_7_1.png}
    \end{center}
    { \hspace*{\fill} \\}
    
    \begin{center}
    \adjustimage{max size={0.9\linewidth}{0.9\paperheight}}{output_7_2.png}
    \end{center}
    { \hspace*{\fill} \\}
    
    \begin{center}
    \adjustimage{max size={0.9\linewidth}{0.9\paperheight}}{output_7_3.png}
    \end{center}
    { \hspace*{\fill} \\}
    
    \begin{center}
    \adjustimage{max size={0.9\linewidth}{0.9\paperheight}}{output_7_4.png}
    \end{center}
    { \hspace*{\fill} \\}
    
    \begin{center}
    \adjustimage{max size={0.9\linewidth}{0.9\paperheight}}{output_7_5.png}
    \end{center}
    { \hspace*{\fill} \\}
    
    As one can see, by taking the Poincare section which consists of an
\(\omega\) vs \(\theta\) plot with one point plotted per drive cycle
(time approximately an integer times 2*pi divided by the natural
frequency of the pendulum), the pendulum undergoes period doubling. In
the period 1 regime, one can confirm that the poincare section only
gives a single unique point (within numerical accuracy).

    \hypertarget{problem-3}{%
\subsection{Problem 3}\label{problem-3}}

\begin{enumerate}
\def\labelenumi{\alph{enumi})}
\item
  For the pendulum, plot x\(_n\) vs \(\mu\) which is the bifurcation
  diagram for the logistic map.
\item
  Estimate the Feigenbaum parameter.
\end{enumerate}

    \begin{Verbatim}[commandchars=\\\{\}]
{\color{incolor}In [{\color{incolor}5}]:} \PY{k+kn}{import} \PY{n+nn}{importlib}
        \PY{k+kn}{import} \PY{n+nn}{Particle1D}
        \PY{k+kn}{from} \PY{n+nn}{Particle1D} \PY{k}{import} \PY{n}{Pendulum}
        \PY{n}{importlib}\PY{o}{.}\PY{n}{reload}\PY{p}{(}\PY{n}{Particle1D}\PY{p}{)}
        \PY{k+kn}{import} \PY{n+nn}{LogisticMap} \PY{k}{as} \PY{n+nn}{lm}
\end{Verbatim}


    \begin{Verbatim}[commandchars=\\\{\}]
{\color{incolor}In [{\color{incolor}6}]:} \PY{c+c1}{\PYZsh{} Generate a bifurcation diagram}
        \PY{n}{mu\PYZus{}start} \PY{o}{=} \PY{l+m+mi}{0}
        \PY{n}{mu\PYZus{}stop} \PY{o}{=} \PY{l+m+mi}{4}
        \PY{n}{mu\PYZus{}steps} \PY{o}{=} \PY{l+m+mi}{10000}
        
        \PY{n}{x0} \PY{o}{=} \PY{l+m+mf}{0.8} 
        
        \PY{c+c1}{\PYZsh{} Generate an array of mu values}
        \PY{n}{mus} \PY{o}{=} \PY{n}{np}\PY{o}{.}\PY{n}{linspace}\PY{p}{(}\PY{n}{mu\PYZus{}start}\PY{p}{,} \PY{n}{mu\PYZus{}stop}\PY{p}{,} \PY{n}{mu\PYZus{}steps}\PY{p}{)}
          
        \PY{c+c1}{\PYZsh{} containers for mu values and logistic map values}
        \PY{n}{mu\PYZus{}list} \PY{o}{=} \PY{p}{[}\PY{p}{]}
        \PY{n}{m\PYZus{}list} \PY{o}{=} \PY{p}{[}\PY{p}{]}
        
        
        \PY{k}{for} \PY{n}{mu} \PY{o+ow}{in} \PY{n}{mus}\PY{p}{:}
            \PY{n}{m} \PY{o}{=} \PY{n}{lm}\PY{o}{.}\PY{n}{LogisticMap}\PY{p}{(}\PY{n}{x0}\PY{p}{,} \PY{n}{mu} \PY{o}{=} \PY{n}{mu}\PY{p}{)}
            
            \PY{c+c1}{\PYZsh{} initialize method removes transient}
            \PY{n}{m}\PY{o}{.}\PY{n}{initialize}\PY{p}{(}\PY{l+m+mi}{100}\PY{p}{)}
            \PY{n}{m}\PY{o}{.}\PY{n}{iterate}\PY{p}{(}\PY{l+m+mi}{1000}\PY{p}{)}
                
            \PY{n}{m\PYZus{}points} \PY{o}{=} \PY{n}{np}\PY{o}{.}\PY{n}{array}\PY{p}{(}\PY{n}{m}\PY{o}{.}\PY{n}{xList}\PY{p}{)}
            \PY{n}{m\PYZus{}points} \PY{o}{=} \PY{n}{np}\PY{o}{.}\PY{n}{unique}\PY{p}{(}\PY{n}{np}\PY{o}{.}\PY{n}{round}\PY{p}{(}\PY{n}{m\PYZus{}points}\PY{p}{,}\PY{l+m+mi}{8}\PY{p}{)}\PY{p}{)}
            
            \PY{k}{for} \PY{n}{m} \PY{o+ow}{in} \PY{n}{m\PYZus{}points}\PY{p}{:}
                \PY{n}{mu\PYZus{}list}\PY{o}{.}\PY{n}{append}\PY{p}{(}\PY{n}{mu}\PY{p}{)}
                \PY{n}{m\PYZus{}list}\PY{o}{.}\PY{n}{append}\PY{p}{(}\PY{n}{m}\PY{p}{)}
                
        \PY{n}{low\PYZus{}bound} \PY{o}{=} \PY{p}{[}\PY{l+m+mf}{0.9}\PY{p}{,} \PY{l+m+mf}{2.8}\PY{p}{,} \PY{l+m+mf}{3.4}\PY{p}{,} \PY{l+m+mf}{3.5}\PY{p}{,} \PY{l+m+mf}{3.56}\PY{p}{,} \PY{l+m+mf}{3.565}\PY{p}{]}
        \PY{n}{up\PYZus{}bound} \PY{o}{=} \PY{p}{[}\PY{l+m+mf}{1.1}\PY{p}{,} \PY{l+m+mf}{3.1}\PY{p}{,} \PY{l+m+mf}{3.5}\PY{p}{,} \PY{l+m+mf}{3.6}\PY{p}{,} \PY{l+m+mf}{3.57}\PY{p}{,} \PY{l+m+mf}{3.58}\PY{p}{]}
        \PY{k}{for} \PY{n}{it} \PY{o+ow}{in} \PY{n+nb}{range}\PY{p}{(}\PY{n+nb}{len}\PY{p}{(}\PY{n}{low\PYZus{}bound}\PY{p}{)}\PY{p}{)}\PY{p}{:}
            \PY{n}{fig} \PY{o}{=} \PY{n}{plt}\PY{o}{.}\PY{n}{figure}\PY{p}{(}\PY{n}{figsize} \PY{o}{=} \PY{p}{[}\PY{l+m+mi}{8}\PY{p}{,}\PY{l+m+mi}{5}\PY{p}{]}\PY{p}{)}
            \PY{n}{ax} \PY{o}{=} \PY{n}{fig}\PY{o}{.}\PY{n}{add\PYZus{}subplot}\PY{p}{(}\PY{l+m+mi}{111}\PY{p}{)}
            \PY{n}{ax}\PY{o}{.}\PY{n}{plot}\PY{p}{(}\PY{n}{mu\PYZus{}list}\PY{p}{,} \PY{n}{m\PYZus{}list}\PY{p}{,}\PY{l+s+s1}{\PYZsq{}}\PY{l+s+s1}{.}\PY{l+s+s1}{\PYZsq{}}\PY{p}{,} \PY{n}{markersize} \PY{o}{=} \PY{l+m+mf}{0.5}\PY{p}{)}
            \PY{n}{ax}\PY{o}{.}\PY{n}{set\PYZus{}xlabel}\PY{p}{(}\PY{l+s+s1}{\PYZsq{}}\PY{l+s+s1}{\PYZdl{}}\PY{l+s+s1}{\PYZbs{}}\PY{l+s+s1}{mu\PYZdl{}}\PY{l+s+s1}{\PYZsq{}}\PY{p}{)}
            \PY{n}{ax}\PY{o}{.}\PY{n}{set\PYZus{}ylabel}\PY{p}{(}\PY{l+s+s1}{\PYZsq{}}\PY{l+s+s1}{\PYZdl{}x\PYZus{}n\PYZdl{}}\PY{l+s+s1}{\PYZsq{}}\PY{p}{)}
            \PY{n}{plt}\PY{o}{.}\PY{n}{xlim}\PY{p}{(}\PY{n}{low\PYZus{}bound}\PY{p}{[}\PY{n}{it}\PY{p}{]}\PY{p}{,}\PY{n}{up\PYZus{}bound}\PY{p}{[}\PY{n}{it}\PY{p}{]}\PY{p}{)}
\end{Verbatim}


    \begin{center}
    \adjustimage{max size={0.9\linewidth}{0.9\paperheight}}{output_11_0.png}
    \end{center}
    { \hspace*{\fill} \\}
    
    \begin{center}
    \adjustimage{max size={0.9\linewidth}{0.9\paperheight}}{output_11_1.png}
    \end{center}
    { \hspace*{\fill} \\}
    
    \begin{center}
    \adjustimage{max size={0.9\linewidth}{0.9\paperheight}}{output_11_2.png}
    \end{center}
    { \hspace*{\fill} \\}
    
    \begin{center}
    \adjustimage{max size={0.9\linewidth}{0.9\paperheight}}{output_11_3.png}
    \end{center}
    { \hspace*{\fill} \\}
    
    \begin{center}
    \adjustimage{max size={0.9\linewidth}{0.9\paperheight}}{output_11_4.png}
    \end{center}
    { \hspace*{\fill} \\}
    
    \begin{center}
    \adjustimage{max size={0.9\linewidth}{0.9\paperheight}}{output_11_5.png}
    \end{center}
    { \hspace*{\fill} \\}
    
    To calculate the Feigenbaum parameter \(\delta\), we use its definition

\[\delta = \lim_{n\to\infty} \frac{\mu_{n}-\mu_{n-1}}{\mu_{n+1}-\mu_{n}}\]

As recommended, this number can be estimated by looking at a finite
number (n = 8) of bifurcations.

    \begin{Verbatim}[commandchars=\\\{\}]
{\color{incolor}In [{\color{incolor}9}]:} \PY{k}{def} \PY{n+nf}{feigen}\PY{p}{(}\PY{n}{a}\PY{p}{,}\PY{n}{b}\PY{p}{,}\PY{n}{c}\PY{p}{)}\PY{p}{:}
            \PY{k}{return} \PY{p}{(}\PY{n}{b}\PY{o}{\PYZhy{}}\PY{n}{a}\PY{p}{)}\PY{o}{/}\PY{p}{(}\PY{n}{c}\PY{o}{\PYZhy{}}\PY{n}{b}\PY{p}{)}
        \PY{n}{mu}\PY{o}{=}\PY{p}{[}\PY{l+m+mf}{1.0}\PY{p}{,} \PY{l+m+mf}{2.97}\PY{p}{,} \PY{l+m+mf}{3.450}\PY{p}{,} \PY{l+m+mf}{3.545}\PY{p}{,} \PY{l+m+mf}{3.565}\PY{p}{]}
        \PY{k}{for} \PY{n}{it} \PY{o+ow}{in} \PY{n+nb}{range}\PY{p}{(}\PY{l+m+mi}{3}\PY{p}{)}\PY{p}{:}
            \PY{n+nb}{print}\PY{p}{(}\PY{n}{feigen}\PY{p}{(}\PY{n}{mu}\PY{p}{[}\PY{l+m+mi}{0}\PY{o}{+}\PY{n}{it}\PY{p}{]}\PY{p}{,} \PY{n}{mu}\PY{p}{[}\PY{l+m+mi}{1}\PY{o}{+}\PY{n}{it}\PY{p}{]}\PY{p}{,} \PY{n}{mu}\PY{p}{[}\PY{l+m+mi}{2}\PY{o}{+}\PY{n}{it}\PY{p}{]}\PY{p}{)}\PY{p}{)}
\end{Verbatim}


    \begin{Verbatim}[commandchars=\\\{\}]
4.104166666666667
5.0526315789473815
4.749999999999983

    \end{Verbatim}

    This value for the feigenbaum parameter \(\delta_{approx} = 4.75\) is
not far off from the literature value \(\delta = 4.669\), so we conclude
that we reproduce the literature value. We can get a better
approximation by increasing the resolution of \(\mu\) or using special
data types with more precision.


    % Add a bibliography block to the postdoc
    
    
    
    \end{document}
